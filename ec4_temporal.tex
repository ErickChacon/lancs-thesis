\chapter{Temporal Modelling}
\thispagestyle{empty}
\label{cha:4_intro}



%Los resultados pueden ser separados en secciones que recuerda que deben estar relacionados a los objetivos y preguntas de investigaci�n planteadas...
%
%\section{Predicci�n de la Batimetr�a del Mar Peruano}
%
%Esta secci�n responde a los objetivos 1 y 2...
%
%El primer resultado que se obtuvo es..
%
%
%\subsection{Comparaci�n del Mapa de Batimetr�a Obtenido con la Batimetr�a del Proyecto SRTM30}
%
%
% Una fuerte desventaja del mapa de SRTM30 ... En la tabla \ref{tab:comp_bathy_srtm} se muestra las principales caracter�sticas entre las dos grillas.
% 
%\begin{table}[H]\footnotesize
%\caption{Comparaci�n del Mapa de Batimetr�a Obtenido con la Batimetr�a del Proyecto SRTM30}
%\centering
%\begin{tabular}{ |l|l|l| }
%\hline
%\textbf{ Caracter�sticas} & \textbf{Mapa batim�trico del estudio} & \textbf{Mapa batim�trico de SRTM-30} \\ \hline \hline
%Resoluci�n & Alta (50 metros) & Baja (1000 metros aprox.) \\ \hline
%Detalle & Mayor detalle en zonas con mayor informaci�n & Menor detalle \\ \hline
%Error de predicci�n & Si tiene & No tiene \\ \hline
%Detalle & Mayor detalle en zonas con mayor informaci�n & Menor detalle \\ \hline
%Informaci�n local & Cuenta con datos de DHN, IRD e IMARPE & No tiene \\ \hline
%Cruceros internacionales & Cuenta con datos de la NOAA & Cuenta con datos de la NOAA \\ \hline 
%\end{tabular}
%\label{tab:comp_bathy_srtm}
%\end{table}
%
%\subsection{Comparaci�n entre un Modelo que no Considera la Diferencia entre Fuentes de Datos y el Modelo Empleado}
%
%La discrepancia entre fuentes de datos no es homog�nea espacialmente...
%
%
%\section{Identificaci�n de Clusters Sedimentarios}
%
%Responde a los siguientes objetivos
%
%\subsection{Clusters Encontrados Seg�n los Perfiles de Sedimentolog�a}
%
%Se encontraron cuatro clusters de perfiles sedimentarios seg�n la granulometr�a y la composici�n de textura de los sedimentos...
%
%\section{Modelamiento de Presencia de Depocentros Sedimentarios}
%
%Responde a los siguientes objetivos...
%
%\subsection{Variables Consideradas y Factores Incorporados al Modelo}
%
%En primer lugar se considera un modelo lineal generalizado para capturar la tendencia explicada por la morfolog�a y batimetr�a del fondo marino...
%
%\section{Predicci�n de la Ubicaci�n de Depocentros Sedimentarios}
%
%Responde al �ltimo objetivo...
%
%\subsection{Caracter�sticas Generales del Mapa de Probabilidades sobre la Presencia de Depocentros}
%
%La zona norte puede...