
\chapter*{Introduction}
\thispagestyle{empty} %not enumerate the first page
\label{cha:0_intro}

%\section{Overview}
Vector-borne diseases are a main world concern, representing $17\%$ of all infectious diseases and more than 1 million deaths annually. These can be transmitted from humans or animals to humans by the so called vectors. Although there are several species of vectors, the most effective vector is the mosquito that become a host after feeding from a infected person and then spread the disease-causing pathogen to uninfected people. The main population affected are those living in tropical region. Particularly, dengue and malaria, transmitted by mosquitoes, are the most sensible vector-borne diseases because the former has the highest incidence grow in the last 50 years (30-fold) with a 2.5 billion people at risk and the second has the highest morbidity incidence  with an estimated of 627 thousand deaths in 2012.~\cite{WorldHealthOrganisation2014a}\\

The presence and incidence of vector-borne diseases are determined by the vector-pathogen-host relationship, where the environment and the socio-economic factors play a main role~\cite{Tabachnick2010}. For the dengue case, the climate (temperature and precipitation), population growth, international travel, poverty and lack of sustained programmes are the main factors of incidence~\cite{Guzman2010, Gubler2006}. Similarly for malaria,
climate (temperature, rainfall, and relative humidity), sustained programmes, social and economic status are major factors~\cite{Huang2011}. However, due to the different behaviour of the vectors, these common factors could have different influences in the incidence of dengue and malaria. \\


% The climate modifies directly the vector, pathogen and host behaviour influencing, mainly, the life expectancy, density population of the vector~\cite{Tabachnick2010}. In Brazil, for instance, it influences the seasonal component of dengue incidence. On the other hand, socio-economic factors can facilitate the transmissions. It is thought that current incidence increase of vector-borne diseases is due to unplanned population growth and the deterioration of housing leading to a lack of basic services (e.g., water, sewer, and waste management)~\cite{Gubler2010}.\\
  



 %That is the reason why Brazil, in particular the brazilian amazon, is of high interest.


%These diseases are commonly found in tropical and sub-tropical regions and places where access to safe drinking-water and sanitation systems is problematic. That is the reason why Brazil, in particular the brazilian amazon, are of high interest.


%Dengue is a mosquito-borne viral disease that has rapidly spread in all regions of WHO in recent years. Dengue virus is transmitted by female mosquitoes mainly of the species Aedes aegypti and, to a lesser extent, A. albopictus. The disease is widespread throughout the tropics, with local variations in risk influenced by rainfall, temperature and unplanned rapid urbanization. There are 4 distinct, but closely related, serotypes of the virus that cause dengue (DEN-1, DEN-2, DEN-3 and DEN-4). Recovery from infection by one provides lifelong immunity against that particular serotype. However, cross-immunity to the other serotypes after recovery is only partial and temporary. Subsequent infections by other serotypes increase the risk of developing severe dengue.









%Malaria is caused by a parasite called Plasmodium, which is transmitted via the bites of infected mosquitoes. In the human body, the parasites multiply in the liver, and then infect red blood cells. Symptoms of malaria include fever, headache, and vomiting, and usually appear between 10 and 15 days after the mosquito bite. If not treated, malaria can quickly become life-threatening by disrupting the blood supply to vital organs. In many parts of the world, the parasites have developed resistance to a number of malaria medicines. Key interventions to control malaria include: prompt and effective treatment with artemisinin-based combination therapies; use of insecticidal nets by people at risk; and indoor residual spraying with insecticide to control the vector mosquitoes.

Brazil, a tropical country with the highest population in Latin America zone, has favourable climate conditions for mosquito vectors being one of the most affected countries by arboviral diseases. For instance,  it is the country with highest incidence of dengue with peaks from December to May because of climate conditions~\cite{Amancio2015}. Specifically, the geographic, climatic and socio-economic features of the Amazon region make of this a susceptible and endemic area of vector-borne diseases. More than $95\%$ of the viruses responsible of vector-borne diseases of the country has been isolated in the Amazon region~\cite{Paula2015}. In addition, while some diseases has been successfully controlled in the rest of Brazil, they are still a main concern in the Amazon basin; this is the case of the malaria disease~\cite{Achcar2011}.\\

% Most of the cases occur from December to May ~\cite{Amancio2015} because of climate conditions.

%Brazil is the largest and most populated country in Latin America, covering >8 million km2 with an estimated 2002 population of 174,632,932 inhabitants. High popula- tion density areas and cities (up to 12,901 inhabitants/km2) are located mainly on the Atlantic Coast. Most of Brazil has a tropical climate; in the southern region, the climate is subtropical. The rainy season is observed in the first sev- eral months of the year, and the average temperature is >20�C (14).

%Currently, Brazil is the leading country in terms of the number of dengue cases reported. Seasonality with most of the cases ocurring from December to May ~\cite{worldwideAmancio2015}.


% Brazilian Amazon
Actions to control malaria and dengue were mainly focus on the reduction of the vector population and in education of the community about these diseases~\cite{Marzochi1994, Barat2006}. For dengue, in Brazil and the Americas, there was a successful program to eradicate the vector which explain the absence of dengue outbreaks between 1923-1981; however, due to the discontinuance of the program, the vector reinvaded the country becoming a today burden~\cite{Figueiredo1996, Gubler2006}. Similarly, between 1930-1940 good results were obtained in the campaign against a main malaria vector in North-eastern Brazil and currently a significant reduction of cases has been reported in the country; nevertheless, the same success was not obtained in the Amazon region~\cite{Coura2006}. Despite the efforts to control dengue and malaria, these vector-borne diseases are still affecting the population and their quality of life; even worse, dengue is showing new trends that could aggravate the situation like hyper-endemicity and increased genetic diversity~\cite{Figueiredo2012}.\\

In this context, the prediction of incidence is a need to improve control programs in order to prevent outbreaks with an efficient distribution of logistics and human resources to the affected zones within a reasonable time. Although, several attempts to predict epidemics has been made, the influence of the risk factors, the spatial variation and time evolution are not still fully understood. In part it this due to the complexity of vector-pathogen-host ecosystem with emerging patterns in dengue disease ~\cite{Paula2015}, but it could be also due to the high computationally cost that is required for sophisticated spatio-temporal models for disease mapping and prediction.\\

In order to determine the main risk factors affecting malaria and dengue incidence in the Brazilian Amazon between 2006-2013 and to propose a predictive spatio-temporal model, Bayesian hierarchical techniques in the framework of latent Gaussian models were used through the novel INLA (Integrated Nested Laplace Approximation) inference approach~\cite{Rue2009}. The area of study covers 310 municipalities of 6 Federative Units and the considered factors include climatic and socio-economic variables in space, time and space-time domains. An additional goal was also to assess the efficiency of the INLA approach in comparison with Markov Chain Monte Carlo (MCMC) methods for spatial models.\\

This dissertation begins by providing an overview of the epidemiology features of dengue and malaria join with a discussion of the literature review of spatio-temporal modelling of dengue and malaria incidence in chapter \ref{cha:1_intro}. An exploratory analysis of the covariates and temporal, spatial and spatio-temporal patterns are presented in chapter \ref{cha:2_intro}. It will then go on to the explanation of the Laplace Approximation and the Bayesian inference in latent Gaussian models through the Integrated Nested Laplace Approximation in chapter \ref{cha:3_intro}. Chapters \ref{cha:4_intro}, \ref{cha:5_intro} and \ref{cha:6_intro} contain the dengue and malaria incidence modelling joint with the comparison of models and diagnostics for the temporal, spatial and spatio-temporal domains respectively. Finally, concluding remarks of the study are presented in chapter \ref{cha:7_intro}.

%Expect to find





%\section{Literature Review}
%\section{Structure of the Thesis}
